\chapter{总结与展望}
随着网络攻击行为日趋复杂,政府、企业以及个人所面临的安全威胁正在飞速增长,如蠕虫病毒、木马后门、僵尸网络、DDOS攻击等,给企业的信息网络造成严重破坏。保障网络安全最大的挑战之一就是及时发现漏洞,而绝大部分安全漏洞和隐患都与主机属性息息相关。此外,识别网络中主机的多维属性还可以帮助网络运营者有效地进行网络管理、网络资源分配、网络服务质量优化。

\section{本文工作总结}

本文主要完成了以下三点工作:

\begin{enumerate}
    \item \textbf{开放环境中的细粒度主机属性发现。}首先以流为单位,提取目标主机发起的TLS会话中的13维协议首部字段特征和3类流统计特征,包括IP协议的跳数、包长、分片标识等字段,TCP协议的传输窗口大小、窗口缩放因子、最大报文长度等字段,TLS协议的版本、扩展长度、密钥算法套件序列等字段以及流的包长序列统计特征、时间序列统计特征、速率统计特征等。然后结合以LightGBM模型为代表的机器学习算法,识别目标主机的操作系统类型及版本、浏览器类型及版本。
    
    \item \textbf{基于加密流原始载荷的主机属性发现。}随着理论的成熟和机器性能的提升,深度学习模型在流量分类领域里的表现越来越出色。通过提取TLS会话中TCP握手包和Client-Hello包的原始流信息,并结合以CNN模型和LSTM模型为代表的表示学习算法,便可在不需要先验知识的前提下,进一步提高复杂网路中主机属性的识别精度。
    
    \item \textbf{大规模网络中的海量主机属性发现。}本文基于Stacking技术,结合以人工特征为基础的LightGBM模型和以原始流信息为基础的深度学习模型,构建了一个用于海量主机属性发现的原型系统。该系统主要包含四个模块。协议识别模块用于在高速网络中检测、解析并识别TLS流量。特征提取模块用于从原始TLS流量中提取所需的人工特征和原始特征。属性识别模块基于stacking技术,综合各学习模型的的检测结果,得到最终的识别信息。数据存储和可视化模块用于识别结果的存储和可视化展示。
\end{enumerate}

\section{下一步的工作}

本文对加密网络中的主机属性发现技术进行了深入分析,利用协议首部字段特征、流统计特征以及原始字节特征,结合统计机器学习模型和神经网络模型,实现了一套具备良好效果的原型系统,但仍然存在以下问题:

\textbf{构建更可信的标注数据集。}本文利用HTTP协议请求报文的User-Agent字段对数据集进行属性标注,标注的正确性依赖于User-Agent字段的真实性。而User-Agent字段容易篡改和伪装,可能一定程度上影响了属性标注方法的准确性,降低了分类器的识别性能。未来将尝试利用TLS透明代理技术或者与大型网络服务商合作,构建更真实可信的标注数据集。

\textbf{提高系统性能。}由于机器学习模型算法效率受限,本文方法的计算成本较高。今后需要从特征和模型角度入手,优化特征集规模,降低机器学习模型复杂度,提高识别效率。

\textbf{扩展主机属性种类。}本文系统目前仅能识别三类主机属性,包括操作系统种类、版本以及浏览器种类等。识别的类别数目较少,未来工作可以引入更多的主机属性,例如设备类型、设备厂商、主机地理位置等等。

\textbf{分析新型协议。}本文的研究对象主要是采用加密协议TLS 1.2版本或更早版本的网络会话,在未来各种新型协议如QUIC协议、TLS 1.3协议、HTTP 2.0协议等技术的普及将会给主机属性识别带来新的挑战。