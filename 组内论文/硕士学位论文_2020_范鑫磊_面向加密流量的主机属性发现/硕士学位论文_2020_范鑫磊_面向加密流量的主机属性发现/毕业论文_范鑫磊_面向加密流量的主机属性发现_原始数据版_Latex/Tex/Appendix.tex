\chapter{中国科学院大学学位论文撰写要求}

学位论文是研究生科研工作成果的集中体现,是评判学位申请者学术水平、授予其学位的主要依据,是科研领域重要的文献资料。根据《科学技术报告、学位论文和学术论文的编写格式》(GB/T 7713-1987)、《学位论文编写规则》(GB/T 7713.1-2006)和《文后参考文献著录规则》(GB7714—87)等国家有关标准,结合中国科学院大学(以下简称“国科大”)的实际情况,特制订本规定。

\section{论文无附录者无需附录部分}

\section{测试公式编号 \texorpdfstring{$\Lambda,\lambda,\theta,\bar{\Lambda},\sqrt{S_{NN}}$}{$\textLambda,\textlambda,\texttheta,\bar{\textLambda},\sqrt{S_{NN}}$}} \label{sec:testmath}

\begin{equation} \label{eq:appedns}
    \adddotsbeforeeqnnum%
    \begin{cases}
        \frac{\partial \rho}{\partial t} + \nabla\cdot(\rho\Vector{V}) = 0\\
        \frac{\partial (\rho\Vector{V})}{\partial t} + \nabla\cdot(\rho\Vector{V}\Vector{V}) = \nabla\cdot\Tensor{\sigma}\\
        \frac{\partial (\rho E)}{\partial t} + \nabla\cdot(\rho E\Vector{V}) = \nabla\cdot(k\nabla T) + \nabla\cdot(\Tensor{\sigma}\cdot\Vector{V})
    \end{cases}
\end{equation}
\begin{equation}
    \adddotsbeforeeqnnum%
    \frac{\partial }{\partial t}\int\limits_{\Omega} u \, \mathrm{d}\Omega + \int\limits_{S} \unitVector{n}\cdot(u\Vector{V}) \, \mathrm{d}S = \dot{\phi}
\end{equation}
\[
    \begin{split}
        \mathcal{L} \{f\}(s) &= \int _{0^{-}}^{\infty} f(t) e^{-st} \, \mathrm{d}t, \ 
        \mathscr{L} \{f\}(s) = \int _{0^{-}}^{\infty} f(t) e^{-st} \, \mathrm{d}t\\
        \mathcal{F} {\bigl (} f(x+x_{0}) {\bigr )} &= \mathcal{F} {\bigl (} f(x) {\bigr )} e^{2\pi i\xi x_{0}}, \ 
        \mathscr{F} {\bigl (} f(x+x_{0}) {\bigr )} = \mathscr{F} {\bigl (} f(x) {\bigr )} e^{2\pi i\xi x_{0}}
    \end{split}
\]

mathtext: $A,F,L,2,3,5,\sigma$, mathnormal: $A,F,L,2,3,5,\sigma$, mathrm: $\mathrm{A,F,L,2,3,5,\sigma}$.

mathbf: $\mathbf{A,F,L,2,3,5,\sigma}$, mathit: $\mathit{A,F,L,2,3,5,\sigma}$, mathsf: $\mathsf{A,F,L,2,3,5,\sigma}$.

mathtt: $\mathtt{A,F,L,2,3,5,\sigma}$, mathfrak: $\mathfrak{A,F,L,2,3,5,\sigma}$, mathbb: $\mathbb{A,F,L,2,3,5,\sigma}$.

mathcal: $\mathcal{A,F,L,2,3,5,\sigma}$, mathscr: $\mathscr{A,F,L,2,3,5,\sigma}$, boldsymbol: $\boldsymbol{A,F,L,2,3,5,\sigma}$.

vector: $\Vector{\sigma, T, a, F, n}$, unitvector: $\unitVector{\sigma, T, a, F, n}$

matrix: $\Matrix{\sigma, T, a, F, n}$, unitmatrix: $\unitMatrix{\sigma, T, a, F, n}$

tensor: $\Tensor{\sigma, T, a, F, n}$, unittensor: $\unitTensor{\sigma, T, a, F, n}$ 


